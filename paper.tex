\documentclass[11pt, a4paper]{article}

\usepackage[utf8]{inputenc}
\usepackage{amsmath, amssymb, amsthm}
\usepackage{graphicx}
\usepackage{geometry}
\usepackage{hyperref}
\usepackage{booktabs}
\usepackage{natbib}

% Page layout
\geometry{margin=1in}

% Title and Author
\title{\textbf{Reliability Is a System Property: The Representation-Level Control Surfaces (RLCS) Paradigm}}
\author{
    Md Arshad \\
    Department of Computer Science, Jamia Millia Islamia \\
    \texttt{https://github.com/Sulkysubject37/resLik}
}
\date{January 12, 2026}

\begin{document}

\maketitle

\begin{abstract}
Deep learning models deployed in open-world environments frequently encounter inputs that violate their training assumptions, leading to silent failures, overconfident predictions, and catastrophic system instability. Current mitigation strategies primarily treat reliability as a property to be \textit{learned} by the model, through robust training, domain adaptation, or epistemic uncertainty estimation. This work argues that this assumption is structurally flawed for safety-critical and high-throughput systems. Reliability is not an emergent property of learning, but a managed property of systems. This paper introduces \textbf{Representation-Level Control Surfaces (RLCS)}, a systems architecture that embeds lightweight, deterministic reliability sensing directly into the latent feature spaces of learned models. RLCS enforces a strict separation between \textit{sensing} (measuring consistency), \textit{signaling} (mapping measures to recommendations), and \textit{acting} (executing decisions), preventing the conflation of statistical inference with execution policy. This work presents \textbf{resLik} (Residual Likelihood Sensor), the reference instantiation of the RLCS sensing layer, alongside companion sensors for temporal and cross-view consistency. Through multi-domain demonstrations in robotics, applied AI, and data systems, this work shows that RLCS enables the construction of ``self-aware'' pipelines that can deterministically detect distribution shifts, temporal shocks, and modal conflicts without requiring model retraining, complex ensemble fusion, or prohibitive computational overhead.
\end{abstract}

\section{Introduction}

\subsection{The Reliability Assumption in Modern Systems}
The widespread deployment of deep learning models in perception, decision-making, and generative tasks relies on a fundamental, often implicit assumption: that the model's output confidence is a reliable proxy for its correctness. In closed-world settings, where the test distribution $P_{\text{test}}(X)$ closely approximates the training distribution $P_{\text{train}}(X)$, this assumption holds reasonably well. Modern neural networks, particularly large transformers and deep convolutional networks, can achieve high accuracy and reasonable calibration on in-distribution data.

However, in deployed real-world environments, ranging from autonomous vehicles navigating rapidly changing weather conditions to financial fraud detection systems processing evolving attack vectors, this assumption breaks down catastrophically. Neural models act as localized approximators of a high-dimensional data manifold. When input data drifts from this manifold, the model does not necessarily fail gracefully or signal ignorance; instead, it often extrapolates, predicting with high confidence on semantically vacuous or adversarial inputs \citep{hendrycks2017baseline}.

This phenomenon of ``silent failure'' represents a critical systems engineering challenge. It is not merely an accuracy problem to be solved with more data; it is an \textit{observability} problem. The system controller (e.g., the path planner in a robot or the transaction router in a bank), which consumes the model's output, has no independent verification of whether the model's internal representation is operating within its validity envelope.

\subsection{Why Learning Alone Cannot Guarantee Reliability}
The dominant research response to this challenge has been to improve the learning process itself. Techniques such as Bayesian Neural Networks \citep{gal2016dropout}, Deep Ensembles \citep{lakshminarayanan2017simple}, and Adversarial Training \citep{madry2018towards} aim to make the model "aware" of its own limitations. While mathematically rigorous and effective in benchmarks, these approaches suffer from two structural limitations in production systems contexts:

\begin{enumerate}
    \item \textbf{Computational Prohibitiveness}: Techniques like Monte Carlo Dropout or Deep Ensembles drastically increase inference latency (often by factors of $10\times$ or more), rendering them unsuitable for high-frequency control loops (e.g., $>100$ Hz robotics) or high-throughput data ingestion.
    \item \textbf{Conflation of Learning and Sensing}: These methods attempt to bake reliability into the synaptic weights of the model itself. This creates a ``black box'' where the specific mechanism of failure, whether it is sensor noise, semantic novelty, or adversarial corruption, is obscured behind a single, opaque softmax probability distribution.
\end{enumerate}

We argue that relying on the learner to self-diagnose is architecturally insufficient. Just as a physical control system uses independent sensors (e.g., tachometers, accelerometers) to validate actuator performance, a neural system requires independent, deterministic observation of its latent representations.

\subsection{Reliability as a System Property}
This paper advances a simple but structural thesis:

Reliability is not a property learned by models, but a property sensed and managed by systems.

Under this view, reliability must be externalized from the learning objective and treated as a first-class system signal. Rather than asking models to self-assess their trustworthiness, systems should independently sense whether latent representations remain consistent with known reference frames, and manage execution accordingly.

We operationalize this thesis through the Representation-Level Control Surfaces (RLCS) paradigm. RLCS introduces an explicit reliability-sensing layer between representation learning and downstream execution. This layer observes latent representations directly, emits interpretable diagnostic signals, and enables deterministic control responses without embedding policy logic into the model itself.

\subsection{Contributions}
This work makes the following contributions:
\begin{enumerate}
    \item \textbf{RLCS Paradigm}: This paper formalizes Representation-Level Control Surfaces as a systems architecture that strictly separates reliability sensing, control signaling, and execution acting.
    \item \textbf{Sensor Taxonomy}: This paper defines a taxonomy of representation-level reliability sensors, Population-Level, Temporal, and Cross-View, each targeting an orthogonal failure mode.
    \item \textbf{Reference Instantiations}: This work presents resLik (a population-level likelihood-consistency sensor), the Temporal Consistency Sensor (TCS), and an Agreement Sensor, each implemented as forward-only, linear-time operations.
    \item \textbf{Composition Rules}: This paper articulates principled rules for composing multiple reliability sensors without fusion, voting, or arbitration.
    \item \textbf{Multi-System Demonstrations}: This work demonstrates RLCS behavior across applied AI pipelines, robotics perception systems, and high-throughput data ingestion workflows.
\end{enumerate}

\subsection{Paper Organization}
The remainder of this paper is structured as follows. Section 2 formalizes the notion of reliability and outlines system-level design requirements. Section 3 introduces the RLCS architecture and compliance criteria. Section 4 presents the RLCS sensor taxonomy, followed by mathematical formulations in Section 5. Section 6 describes the control surface design, and Section 7 details sensor composition rules. Section 8 demonstrates RLCS behavior across multiple system domains. Limitations and failure modes are discussed in Section 9, followed by related work in Section 10. We conclude with broader implications and future directions in Sections 11 and 12.

\section{Problem Formulation and Design Goals}
This section formalizes what is meant by reliability in the RLCS paradigm and derives the system-level constraints that any representation-level reliability mechanism must satisfy. We also explicitly delineate the boundaries of the paradigm to prevent misinterpretation.

\subsection{What Reliability Is (and Is Not)}
In the context of RLCS, reliability is defined as the statistical consistency of a latent representation with respect to a validated reference frame. The reference frame may correspond to a global training population, a local temporal history, or an independent peer representation.

Crucially, reliability in this sense is a property of representation validity, not task performance. To avoid conceptual overlap with existing terminology, this work distinguishes this definition from several related but non-equivalent concepts:

\begin{itemize}
    \item \textbf{Reliability $\neq$ Accuracy}: A model may operate on a representation that is statistically consistent with its training distribution and still produce an incorrect output due to class ambiguity, label noise, or task complexity. RLCS validates the appropriateness of the input evidence, not the correctness of the prediction. It guarantees that the system is asking a valid question, not that it will produce the correct answer.
    \item \textbf{Reliability $\neq$ Uncertainty}: Uncertainty typically refers to probabilistic measures over model outputs, encompassing epistemic and aleatoric components. RLCS does not attempt to estimate uncertainty in this sense. Instead, it deterministically measures deviations in latent space relative to explicit reference statistics. As a result, RLCS produces interpretable diagnostics rather than probabilistic beliefs.
    \item \textbf{Reliability $\neq$ Calibration}: Calibration concerns the alignment between predicted probabilities and empirical outcome frequencies (e.g., a prediction of 0.7 being correct 70\% of the time). RLCS operates prior to output generation and does not depend on labeled outcomes. It evaluates whether the internal representation itself lies within a regime where calibrated predictions are even meaningful.
\end{itemize}

By construction, RLCS treats reliability as an observability problem: can the system determine, at runtime, whether its internal state remains within a validated operating envelope?

\subsection{System-Level Requirements}
For representation-level reliability sensing to function as a control surface in real systems, it must satisfy a set of engineering constraints that are often secondary in purely algorithmic research but dominant in production environments.

\begin{enumerate}
    \item \textbf{Observability}: The mechanism must expose why a representation is deemed unreliable. Distinct failure modes, such as abrupt temporal discontinuities versus gradual distributional drift, must be separable, as they imply different downstream responses.
    \item \textbf{Interpretability}: Reliability signals must be discrete and semantically meaningful (e.g., PROCEED, DOWNWEIGHT, DEFER, ABSTAIN). Continuous scores alone are insufficient for deterministic system logic, safety audits, and post hoc analysis.
    \item \textbf{Composability}: Reliability mechanisms must be modular. Introducing an additional sensor (e.g., temporal coherence) should not require retraining the encoder, modifying existing sensors, or restructuring the pipeline. Each sensor must operate independently.
    \item \textbf{Determinism}: Given fixed inputs and fixed parameters, the sensing and signaling process must be deterministic. Stochastic sampling, randomized inference, or non-reproducible behavior is incompatible with safety interlocks and audit requirements.
    \item \textbf{Runtime Efficiency}: The computational cost of reliability sensing must be negligible relative to representation extraction. RLCS sensors are therefore constrained to operations linear in the representation dimension ($O(d)$), avoiding covariance estimation or quadratic interactions that scale poorly in high-dimensional latent spaces.
\end{enumerate}

Together, these requirements imply that reliability sensing must be \textbf{cheap, explicit, and externally controllable}, rather than statistically optimal in isolation.

\subsection{Explicit Non-Goals}
To preserve architectural clarity and avoid scope creep, RLCS explicitly excludes the following objectives:

\begin{itemize}
    \item \textbf{End-to-End Learning}: The sensing layer is not optimized via backpropagation during deployment. RLCS sensors are fixed functions whose parameters are configured offline. This design choice ensures predictability and decouples sensing from task optimization.
    \item \textbf{Automatic Correction or Reconstruction}: RLCS detects unreliable representations but does not attempt to repair, denoise, or hallucinate corrected inputs. Automatic correction conflates sensing with acting and introduces unobservable failure modes.
    \item \textbf{Controller or Policy Design}: RLCS does not prescribe how a system should respond to reliability signals. Control policies, such as braking, rerouting, deferring to a human operator, or triggering retraining, remain entirely external to the paradigm.
\end{itemize}

These exclusions are not limitations but deliberate design decisions. RLCS is concerned with making reliability visible, not with resolving uncertainty or optimizing behavior.

\section{Representation-Level Control Surfaces (RLCS)}

\subsection{Architectural Overview}
RLCS introduces an explicit reliability observability layer between representation learning and execution. The architecture enforces a unidirectional flow of information and a strict separation of concerns.

This paper models an RLCS-compliant system as a tuple:
\begin{equation}
    (\mathcal{E}, \mathcal{S}, \Pi, \mathcal{C})
\end{equation}

with the following functional composition:
\begin{equation}
    x \xrightarrow{\mathcal{E}} z \xrightarrow{\mathcal{S}} d \xrightarrow{\Pi} u \xrightarrow{\mathcal{C}} a
\end{equation}

where:
\begin{enumerate}
    \item \textbf{The Encoder ($\mathcal{E}: \mathcal{X} \to \mathbb{R}^d$)}: Maps raw input $x$ to a latent representation $z$. The encoder is treated as a ``state estimator'' that is frozen or updated independently.
    \item \textbf{The Sensor Array ($\mathcal{S}: \mathbb{R}^d \to \mathbb{R}^k$)}: A set of independent functions that measure the consistency of $z$ against various reference frames (population, history, peers). It outputs raw diagnostics $d$.
    \item \textbf{The Control Surface ($\Pi: \mathbb{R}^k \to \mathcal{U}$)}: A stateless, deterministic logic layer that maps diagnostics $d$ to a formal control signal $u \in \mathcal{U}$ (e.g., \texttt{PROCEED}, \texttt{DEFER}, \texttt{ABSTAIN}).
    \item \textbf{The Controller ($\mathcal{C}: \mathcal{U} \times \mathcal{K} \to \mathcal{A}$)}: The external system logic that consumes $u$ alongside system context $k \in \mathcal{K}$ to execute action $a \in \mathcal{A}$.
\end{enumerate}

\begin{figure}[h]
    \centering
    \includegraphics[width=\textwidth]{figures/fig1_architecture.pdf}
    \caption{Schematic overview of the RLCS architecture, illustrating the separation between representation learning (encoder), reliability sensing (sensor array), control signaling (control surface), and external execution (controller).}
    \label{fig:architecture}
\end{figure}

\subsection{RLCS Sensor Array}
The sensor array $\mathcal{S}$ consists of a set of independent functions:
\begin{equation}
    \mathcal{S} = \{ s_1, s_2, \dots, s_k \}, \quad s_i: \mathbb{R}^d \rightarrow \mathbb{R}
\end{equation}

Each sensor observes the same latent representation $z$, but evaluates it against a distinct reference frame: population statistics, temporal history, or peer representations. The output of the sensor array is a diagnostic vector:
\begin{equation}
    d = \mathcal{S}(z) = \big[ s_1(z), s_2(z), \dots, s_k(z) \big]
\end{equation}

Key properties of RLCS sensors:
\begin{itemize}
    \item Forward-only: no gradients are required or propagated.
    \item Linear-time: each $s_i$ operates in $O(d)$.
    \item Semantically interpretable: each diagnostic corresponds to a specific failure mode.
    \item Orthogonal by construction: sensors do not modify or condition each other.
\end{itemize}

Importantly, sensors do not alter the representation $z$. They observe; \textbf{they do not intervene}.

\subsection{Control Surface: From Diagnostics to Signals}
The control surface $\Pi$ is a deterministic mapping from diagnostics to a finite signal set:
\begin{equation}
    \Pi: \mathbb{R}^k \rightarrow \mathcal{U}
\end{equation}

where:
\begin{equation}
    \mathcal{U} = \{ \texttt{PROCEED}, \texttt{DOWNWEIGHT}, \texttt{DEFER}, \texttt{ABSTAIN} \}
\end{equation}

Unlike probabilistic thresholds or learned policies, $\Pi$ is explicitly rule-based. A typical instantiation takes the form:
\begin{equation}
    u =
    \begin{cases}
    \texttt{ABSTAIN}, & \exists i : s_i(z) > \tau_i^{\text{hard}} \\
    \texttt{DEFER}, & \exists i : s_i(z) > \tau_i^{\text{soft}} \\
    \texttt{DOWNWEIGHT}, & \text{marginal violations} \\
    \texttt{PROCEED}, & \text{otherwise}
    \end{cases}
\end{equation}

where $\tau_i^{\text{hard}}$ and $\tau_i^{\text{soft}}$ are sensor-specific thresholds. This mapping ensures reproducibility, auditability, and explicit risk tolerance configuration. The control surface does not arbitrate truth between sensors; it encodes conservative system policy.

\subsection{External Controller and Responsibility Boundary}
The external controller $\mathcal{C}$ consumes the control signal $u$ alongside system context $k \in \mathcal{K}$ to execute an action:
\begin{equation}
    a = \mathcal{C}(u, k)
\end{equation}

Examples include braking or slowing a robot, routing data to a fallback model, deferring to human review, or tagging samples for retraining. RLCS deliberately excludes $\mathcal{C}$ from its scope. This boundary enforces clear responsibility separation:
\begin{itemize}
    \item RLCS answers: ``Is this representation reliable?''
    \item The controller answers: ``What should we do about it?''
\end{itemize}

\subsection{RLCS Compliance Criteria}
A system is considered RLCS-compliant if it satisfies the following invariants:
\begin{enumerate}
    \item \textbf{Passive Sensing}: Sensors observe representations but do not modify them for downstream tasks.
    \item \textbf{Non-Executive Operation}: No sensor or control surface component executes side effects.
    \item \textbf{Bounded State}: Sensors may retain minimal bounded state (e.g., $z_{t-1}$) but no adaptive memory.
    \item \textbf{No Arbitration or Fusion}: Sensor outputs are not averaged, voted on, or fused into a single scalar.
    \item \textbf{Deterministic Semantics}: Identical inputs produce identical diagnostics and signals.
\end{enumerate}

Violating any of these conditions collapses RLCS into either a learning algorithm or a controller, breaking the paradigm.

\subsection{Distinction from Related Architectures}
RLCS is structurally distinct from several adjacent approaches:
\begin{itemize}
    \item \textbf{Ensemble Learning}: Ensembles aggregate predictions to improve accuracy. RLCS does not aggregate predictions; it observes representation validity.
    \item \textbf{Uncertainty Estimation}: Probabilistic uncertainty quantifies belief over outputs. RLCS emits deterministic diagnostics tied to explicit reference frames.
    \item \textbf{Runtime Monitoring}: Traditional monitors log alerts asynchronously. RLCS sits on the execution path and gates system behavior in real time.
    \item \textbf{Control Policies}: RLCS does not decide actions. It provides evidence to systems that do.
\end{itemize}

These distinctions are architectural, not merely implementation choices. The mathematical instantiations of these sensors are defined in Section 5.

\section{RLCS Sensor Taxonomy}

Reliability at the representation level is inherently multi-dimensional. No single diagnostic can distinguish all failure modes encountered in open-world systems. RLCS therefore formalizes sensors as first-class, independent observers of latent representations, each defined by the reference frame against which consistency is evaluated.

This section introduces a taxonomy of RLCS sensors based on their reference frame and the class of failure modes they detect.

\subsection{Sensor as First Class Observers}
An RLCS sensor is a deterministic function $s: \mathbb{R}^d \rightarrow \mathbb{R}$ that evaluates a latent representation $z$ against a specific reference frame. The defining property of an RLCS sensor is not its mathematical form, but its epistemic role: each sensor answers a different question about representation validity.

Crucially, sensors do not predict labels, do not estimate correctness, and do not modify representations. They answer only: Is this representation consistent with respect to a particular notion of validity? Different notions of validity correspond to different reference frames.

\subsection{Taxonomy by Reference Frame}
This paper formalizes three classes of RLCS sensors, each targeting an orthogonal failure mode. Together, they span the dominant sources of representation unreliability in deployed systems.

\begin{figure}[h]
    \centering
    \includegraphics[width=0.8\textwidth]{figures/fig2_taxonomy.pdf}
    \caption{RLCS sensor taxonomy. Each sensor observes the same latent representation but evaluates it against a different reference frame: global population statistics, local temporal history, or an independent peer representation. Orthogonality arises from the reference frame, not from the implementation.}
    \label{fig:taxonomy}
\end{figure}

\begin{table}[h]
    \centering
    \begin{tabular}{@{}llll@{}}
        \toprule
        \textbf{Sensor Class} & \textbf{Reference Frame} & \textbf{Primary Failure Mode} & \textbf{Example} \\
        \midrule
        Population-Level & Global Training Set & Out-of-Distribution (OOD) & \textbf{resLik} \\
        Temporal & Local History ($t-1$) & Shock, Glitch, Instability & \textbf{TCS} \\
        Cross-View & Peer Representation & Modal Conflict, Sensor Failure & \textbf{Agreement} \\
        \bottomrule
    \end{tabular}
    \caption{Summary of RLCS Sensor Types.}
    \label{tab:sensor_taxonomy}
\end{table}

\subsection{Population-Level Consistency: ResLik}
Population-level sensors evaluate whether a representation is statistically consistent with a validated reference population, typically derived from training data.

\textbf{Definition}: Let $\mathcal{P}$ denote a reference population in latent space, summarized by fixed statistics. A population-level sensor evaluates the deviation of $z$ from $\mathcal{P}$.

\textbf{Failure Modes Detected}: Out-of-distribution inputs, semantic novelty, adversarial or corrupted samples.

\textbf{Representative Instantiation: ResLik}: The Residual Likelihood Sensor (ResLik) measures normalized deviation of $z$ from population statistics. It answers the question: \textit{Does this representation plausibly belong to the same distribution as the training data?}

\textbf{Blind Spot}: Population-level sensors cannot distinguish valid novelty (e.g., concept drift) from invalid corruption. Both appear as deviation from the reference distribution.

\subsection{Temporal Consistency: TCS}
Temporal sensors evaluate whether a representation evolves coherently over time.

\textbf{Definition}: Let $z_t$ and $z_{t-1}$ denote successive latent representations. A temporal sensor measures the magnitude and structure of change between them, normalized to ensure scale invariance.

\textbf{Failure Modes Detected}: Sudden shocks or glitches, sensor dropouts, physically implausible state transitions.

\textbf{Representative Instantiation: TCS}: The Temporal Consistency Sensor (TCS) evaluates short-horizon self-consistency. It answers the question: \textit{Is the system changing faster or more abruptly than expected?}

\textbf{Blind Spot}: Temporal sensors are insensitive to slow, monotonic drift. A representation can remain temporally smooth while gradually departing from the valid population manifold.

\subsection{Cross-View Consistency: Agreement Sensor}
Cross-view sensors evaluate agreement between independent representations of the same underlying input.

\textbf{Definition}: Let $z^{(1)}$ and $z^{(2)}$ denote representations derived from distinct modalities, models, or pipelines. A cross-view sensor measures their geometric or semantic alignment.

\textbf{Failure Modes Detected}: Modal disagreement, partial sensor failure, ambiguous or conflicting evidence.

\textbf{Representative Instantiation: Agreement Sensor}: The Agreement Sensor measures alignment (e.g., cosine similarity) between peer representations. It answers the question: \textit{Do independent observers agree on the state of the world?}

\textbf{Blind Spot}: Cross-view sensors cannot detect common-mode failures, where multiple sensors fail in the same correlated manner.

\subsection{Orthogonality and Complementarity}
The strength of the RLCS sensor taxonomy lies in orthogonality by reference frame:
\begin{itemize}
    \item Population-level sensors compare against what was seen before.
    \item Temporal sensors compare against what just happened.
    \item Cross-view sensors compare against what another observer sees.
\end{itemize}

Because each sensor encodes a distinct notion of validity, their signals are complementary rather than redundant. This orthogonality enables conservative, interpretable system behavior when sensors are composed, without resorting to fusion or arbitration.

\section{Mathematical Formulation}

This section presents the mathematical formulation of the RLCS sensor family. All sensors are designed as forward-only, deterministic functions over latent representations, with bounded computational cost and interpretable semantics. The emphasis is not on probabilistic optimality, but on structural reliability observability.

Throughout, let $z \in \mathbb{R}^d$ denote a latent representation produced by a fixed encoder.

\subsection{Shared Design Properties}
All RLCS sensors satisfy the following mathematical constraints:
\begin{enumerate}
    \item \textbf{Forward-Only Evaluation}: Each sensor is a fixed function $s: \mathbb{R}^d \rightarrow \mathbb{R}$ evaluated at inference time. No gradients, sampling, or optimization are involved.
    \item \textbf{Monotonicity}: Sensor outputs are monotonic with respect to the magnitude of deviation from their reference frame. Larger deviations produce larger discrepancy scores or lower consistency scores.
    \item \textbf{Scale Awareness}: Sensors normalize raw deviations to avoid coupling diagnostic magnitude to the absolute scale of the representation.
    \item \textbf{Linear-Time Complexity}: All operations scale linearly with representation dimension $d$ ($O(d)$), ensuring negligible overhead relative to encoding.
\end{enumerate}

These properties ensure that sensors are predictable, composable, and suitable for real-time systems.

\subsection{ResLik Formulation}
ResLik computes a normalized discrepancy score $D_{\text{pop}}$ and a gating factor $g$. Given input $z \in \mathbb{R}^d$ and frozen reference statistics $\mu, \sigma \in \mathbb{R}^d$:

\begin{equation}
    \tilde{z}_i = \frac{z_i - \mu_i}{\sigma_i + \epsilon}, \quad i = 1, \dots, d
\end{equation}

The discrepancy is the mean absolute deviation (Z-score equivalent) across dimensions:
\begin{equation}
    D_{\text{pop}} = \frac{1}{d} \sum_{i=1}^d |\tilde{z}_i|
\end{equation}

The gating factor applies a soft penalty based on this discrepancy, with a learnable sensitivity $\lambda$ and a ``dead-zone'' threshold $\tau$ to ignore benign variance:
\begin{equation}
    g = \exp\left(-\lambda \cdot \max(0, D_{\text{pop}} - \tau)\right)
\end{equation}

\begin{figure}[h]
    \centering
    \includegraphics[width=0.6\textwidth]{figures/fig3_reslik_response.pdf}
    \caption{Behavioral response of the ResLik gating function.}
    \label{fig:reslik_response}
\end{figure}

\subsection{Temporal Consistency Formulation}
TCS computes a normalized drift score $D_{\text{time}}$ based on the Euclidean distance between time steps. To ensure scale invariance, the drift is normalized by the magnitude of the previous state:

\begin{equation}
    \Delta_t = \| z_t - z_{t-1} \|_2
\end{equation}
\begin{equation}
    D_{\text{time}} = \frac{\Delta_t}{\| z_{t-1} \|_2 + \epsilon}
\end{equation}

The temporal consistency score $T_{\text{consistency}}$ is mapped monotonically to $[0, 1]$:
\begin{equation}
    T_{\text{consistency}} = \exp\left(-\alpha \cdot D_{\text{time}}\right)
\end{equation}
where $\alpha$ is a sensitivity hyperparameter.

\subsection{Agreement Formulation}
The Agreement Sensor evaluates consistency between two independent representations of the same input. Let $z^{(1)}, z^{(2)} \in \mathbb{R}^d$ denote peer representations. Agreement is measured via cosine similarity:
\begin{equation}
    A(z^{(1)}, z^{(2)}) = \frac{ z^{(1)} \cdot z^{(2)} } { \| z^{(1)} \|_2 \, \| z^{(2)} \|_2 + \epsilon }
\end{equation}

This yields a bounded score $A \in [-1, 1]$, where higher values indicate stronger alignment. The disagreement metric is defined as:
\begin{equation}
    D_{\text{agree}} = 1 - A
\end{equation}
which maps perfect agreement to zero and increasing misalignment to larger values.

\subsection{Unified Diagnostic Vector}
For a system employing all three sensors, the diagnostic output takes the form:
\begin{equation}
    d(z) = \big[ D_{\text{pop}}(z),\; D_{\text{time}}(z_t),\; D_{\text{agree}}(z^{(1)}, z^{(2)}) \big]
\end{equation}

Each component corresponds to a distinct notion of validity and must be interpreted independently. RLCS explicitly forbids collapsing this vector into a single scalar score.

The diagnostic vector is consumed by the control surface, described in Section 6.

\section{Control Surface Design}

\subsection{From Diagnostics to Control Signals}
The control surface $\Pi$ is defined as a deterministic mapping:
\begin{equation}
    \Pi : \mathbb{R}^k \rightarrow \mathcal{U}
\end{equation}
where $\mathcal{U}$ is a finite, semantically interpretable signal set. Given a diagnostic vector $d(z) = [D_{\text{pop}}, D_{\text{time}}, D_{\text{agree}}]$, the role of $\Pi$ is to compress continuous, high-dimensional evidence into a discrete control recommendation.

The canonical RLCS signal set is defined as:
\begin{itemize}
    \item \textbf{PROCEED}: All diagnostics lie within nominal bounds.
    \item \textbf{DOWNWEIGHT}: Diagnostics indicate marginal inconsistency. The representation may be usable, but its influence should be reduced.
    \item \textbf{DEFER}: Reliability is insufficient for autonomous execution. The representation should be routed to a fallback mechanism.
    \item \textbf{ABSTAIN}: One or more diagnostics indicate a hard violation (e.g., shock, corruption). The representation is invalid and should be discarded.
\end{itemize}

These signals are \textbf{recommendations}, not commands. Their semantics are intentionally abstract, allowing different systems to map them to context-appropriate actions.

\subsection{Deterministic Logic and Thresholding}
The defining property of the control surface is \textbf{determinism}. A typical control surface instantiation is expressed as a monotonic rule set:
\begin{equation}
    \Pi(d) =
    \begin{cases}
    \texttt{ABSTAIN}, & \exists i : d_i > \tau_i^{\text{hard}} \\
    \texttt{DEFER}, & \exists i : d_i > \tau_i^{\text{soft}} \\
    \texttt{DOWNWEIGHT}, & \text{marginal violations} \\
    \texttt{PROCEED}, & \text{otherwise}
    \end{cases}
\end{equation}

where $\tau_i^{\text{hard}}$ and $\tau_i^{\text{soft}}$ are sensor-specific thresholds. Key properties include monotonicity, explicit risk encoding, and reproducibility.

\subsection{Auditability and Responsibility Separation}
Because the control surface is deterministic and rule-based, it provides a transparent audit trail. Responsibility is explicitly partitioned: RLCS answers ``Is this representation reliable?'' while the controller answers ``What should be done given that assessment?'' This separation prevents hidden policy decisions from being embedded inside statistical models.

\subsection{Control Surface vs. Controllers}
Although the control surface emits signals that influence behavior, it is not a controller in the control-theoretic sense. Controllers map state to actions; control surfaces map diagnostics to recommendations. By keeping control logic external, RLCS ensures that reliability sensing remains general and reusable.

\section{Sensor Composition Rules}

A critical innovation of RLCS is the rigorous definition of how multiple sensors coexist. Naive approaches often use weighted averaging or majority voting, which we identify as anti-patterns.

\subsection{The Independence Invariant}
RLCS enforces that \textbf{sensors are parallel observers, not competing predictors.}
\begin{itemize}
    \item ResLik does not ``know'' about TCS.
    \item TCS does not ``correct'' Agreement.
    \item Each sensor provides an orthogonal slice of state space validity.
\end{itemize}

\subsection{The Non-Arbitration Rule}
The Control Surface must not arbitrate ``truth'' between sensors (e.g., ``ResLik is probably right, so ignore TCS''). Instead, it applies a conservative logic: \textbf{ambiguity is a signal in itself.}

\textbf{Valid Composition (The Conservative OR)}:
\begin{equation}
    \text{State} = \begin{cases} 
    \text{UNRELIABLE} & \text{if } \exists s \in \mathcal{S} : s(z) \text{ reports FAILURE} \\
    \text{RELIABLE} & \text{otherwise}
    \end{cases}
\end{equation}

\subsection{Anti-Patterns}
\begin{itemize}
    \item \textbf{Fusion}: $z_{\text{final}} = w_1 z_1 + w_2 z_2$. This destroys the specific evidence of conflict.
    \item \textbf{Smoothing}: Applying Kalman filters \textit{inside} the sensor. This hides volatility from the controller. Smoothing is an actuator function, not a sensor function.
    \item \textbf{Majority Voting}: Ignoring a sensor because it is the minority outlier. In safety systems, the outlier is often the only one seeing the crash.
\end{itemize}

\section{Multi-System Demonstrations}

This work validated the RLCS paradigm by implementing reference controllers across three distinct domains using the \texttt{resLik} library.

\begin{figure}[h]
    \centering
    \includegraphics[width=\textwidth]{figures/fig4_multisensor.pdf}
    \caption{Multi-Sensor Response to Different Failure Modes. A simulated scenario shows how ResLik detects gradual drift while TCS remains stable, and how TCS detects sudden shock independently. Agreement detects sensor conflict independently.}
    \label{fig:multisensor}
\end{figure}

\subsection{Applied AI Pipelines}
This work first considers an applied AI pipeline where downstream components assume upstream embeddings are semantically valid. In a simulated retrieval stream, the system processed nominal inputs, then gradual domain shift, and finally abrupt corruption. \textbf{ResLik} reported increasing discrepancy under drift, while \textbf{TCS} identified sudden noise injection as incoherent (ABSTAIN). Standard uncertainty estimates typically conflate these cases, whereas RLCS distinguishes valid novelty from invalid corruption.

\subsection{Robotics Perception}
This work next evaluates RLCS in a robotics perception stack with redundant Lidar and camera-based encoders. Under nominal conditions, modalities produce consistent representations. A simulated ``sensor blinding'' event degraded the Lidar signal. The \textbf{Agreement Sensor} detected divergence, while \textbf{ResLik} identified the Lidar representation as OOD, triggering a DEFER signal specifically for that stream.

\subsection{Data Systems}
Finally, this work considers a high-throughput ingestion pipeline. In a continuous stream, corruption triggered simultaneous failures in ResLik and TCS, while concept drift triggered ResLik failure but TCS pass. This allowed the system to automatically tag drifting data for retraining while dropping corrupted data (ABSTAIN), automating data hygiene without human intervention.

\subsection{Summary of Demonstrated Behavior}
Across all domains, RLCS exhibits three consistent properties: failure-mode separation, architectural generality, and non-intrusive integration. Reliability is made explicit, interpretable, and actionable at the system level without entangling it with model optimization.

\section{Limitations and Failure Modes}
RLCS improves the observability of reliability at the representation level, but it does not eliminate uncertainty or failure. The paradigm is intentionally constrained, and its limitations follow directly from its design principles.

\subsection{Dependence on Reference Quality}
Population-level sensors such as ResLik rely on fixed reference statistics ($\mu, \sigma$) derived from a validated dataset. If this reference set is poorly curated, containing systematic outliers, mislabeled data, or unrepresentative samples, the resulting diagnostics will be correspondingly degraded. In such cases, population deviations may be underreported, novelty detection may become desensitized, and corruption may be misclassified as benign variation. This limitation is structural rather than algorithmic: RLCS does not infer reference validity online. Reference construction and maintenance remain a system responsibility.

\subsection{Forward-only Constraint}
The current RLCS formulation is forward-only and non-differentiable. Sensors are evaluated strictly at inference time and do not participate in gradient-based optimization. As a result, RLCS cannot directly regularize representation learning during training, reliability signals cannot be backpropagated into the encoder, and adaptation to new operating regimes requires external updates. This constraint is deliberate. It preserves determinism, auditability, and separation of concerns.

\subsection{Ambiguity is Exposed, Not Resolved}
When sensors emit conflicting diagnostics, e.g., strong population consistency alongside low cross-view agreement, RLCS does not attempt to resolve the conflict. Instead, the conflict itself is surfaced as a reliability signal, the control surface emits a conservative recommendation, and resolution is delegated to the external controller. This behavior is not a deficiency but a design choice. Attempting to arbitrate between sensors would embed hidden policy logic into the sensing layer, undermining interpretability and auditability.

\subsection{Coverage is Not Exhaustive}
RLCS sensors are designed to detect classes of failure, not all possible failures. Certain conditions remain outside their scope, including common-mode failures affecting all modalities identically, semantically incorrect but statistically consistent representations, and adversarial manipulations that preserve latent statistics. RLCS does not claim completeness. It provides structured observability, not guarantees of correctness.

\subsection{Summary}
RLCS makes reliability visible and actionable, but it does not validate task correctness, replace robust training, or eliminate the need for system-level policy design. Its role is diagnostic, not corrective. \textbf{RLCS exposes reliability; it does not resolve ambiguity.} By shifting uncertainty handling from implicit model behavior to explicit system logic, RLCS enables safer, more accountable deployment of learned components.

\section{Related Work}
RLCS intersects with several active research areas spanning machine learning, uncertainty estimation, and safety-critical systems. Rather than competing directly with these approaches, RLCS is best understood as an architectural complement that reframes reliability as a system-level observability problem.

\subsection{Uncertainty Estimation and Model Confidence}
A substantial body of work treats reliability as a form of uncertainty estimation learned by the model itself. Bayesian Neural Networks and approximate Bayesian methods such as Monte Carlo Dropout \citep{gal2016dropout} aim to capture epistemic uncertainty through stochastic inference. Similarly, Deep Ensembles \citep{lakshminarayanan2017simple} estimate predictive uncertainty by aggregating multiple independently trained models.

While effective in controlled settings, these approaches incur significant computational overhead at inference time and conflate uncertainty estimation with prediction. More importantly, they expose uncertainty only at the output layer, offering limited insight into why a model is uncertain or whether its internal representation is valid. RLCS differs fundamentally in both scope and intent. It does not estimate probabilistic uncertainty, nor does it attempt to approximate Bayesian inference. Instead, it deterministically measures consistency of latent representations relative to explicit reference frames.

\subsection{Out-of-Distribution (OOD) Detection}
OOD detection has been extensively studied as a classification or scoring problem. Early baselines relied on softmax confidence \citep{hendrycks2017baseline}, while later approaches introduced input perturbations (ODIN; \citealp{liang2018enhancing}) or distance-based methods using Mahalanobis statistics in feature space \citep{lee2018simple}.

ResLik shares mathematical similarities with feature-space OOD detectors, particularly in its use of normalized deviations from reference statistics. However, the framing is deliberately different. Traditional OOD methods aim to answer the question ``Is this input out-of-distribution?'' as a binary or probabilistic classification task. RLCS reframes this question as ``Should the system act on this representation?'' Population-level deviation is treated as \textbf{diagnostic evidence}, not a final decision. Moreover, RLCS explicitly supports multiple orthogonal sensors (temporal, cross-view), whereas most OOD methods operate in isolation.

\subsection{Representation and Feature Space Analysis}
Several works have explored monitoring internal representations for model introspection or failure detection. Influence functions, activation statistics, and layer-wise diagnostics have been proposed as tools for debugging and interpretability \citep{koh2017understanding, morcos2018insights}. These approaches are typically retrospective and analytic rather than operational. They are used offline to understand model behavior, not as real-time interlocks in deployed systems. RLCS differs by treating representation monitoring as a first-class runtime component, explicitly designed to sit on the execution path with bounded latency.

\subsection{Runtime Monitoring and ML System Safety}
In production ML systems, reliability is often handled through external monitoring infrastructure, dashboards, alerts, and logging pipelines \citep{breck2017ml}. While essential, these mechanisms are typically asynchronous and do not gate execution directly. RLCS occupies a different position: it provides inline observability. Sensors operate synchronously on representations and can influence system behavior before downstream components consume unreliable data. This distinction aligns RLCS more closely with safety interlocks than with post-hoc monitoring.

\subsection{Control-Theoretic Inspirations}
The architectural philosophy of RLCS draws inspiration from control theory, particularly \textbf{Simplex Architectures} for safety-critical systems \citep{sha2001using}. In such systems, a complex, high-performance controller is supervised by a simpler, verified safety controller that intervenes when operating assumptions are violated. RLCS adopts a similar separation of concerns, but applies it to the latent space of learned models. Rather than overriding actions directly, RLCS supervises representations and emits control-relevant signals that external controllers can act upon. Unlike classical control systems, however, RLCS does not require a formal plant model or stability guarantees. Its role is diagnostic rather than corrective.

\subsection{Positioning Summary}
RLCS is not a replacement for uncertainty estimation, a new OOD detection algorithm, or a control policy. Instead, it provides a systems architecture that unifies latent-space diagnostics, deterministic signaling, and external control. By explicitly separating sensing from acting, RLCS complements existing learning-centric approaches while addressing a gap that they do not target directly: runtime reliability observability.

\section{Discussion and Outlook}
The RLCS paradigm reframes reliability from a learned artifact of model training into an explicit, system-managed property. By treating latent representations as observable signals rather than opaque intermediates, RLCS shifts the focus of robustness from parameter tuning to architectural design. This shift aligns ML deployment more closely with established principles in systems and control engineering, where observability, separation of concerns, and determinism are foundational.

A key implication of RLCS is that graceful failure becomes a design choice rather than an emergent accident. Systems equipped with representation-level sensing can distinguish between different classes of invalidity, novelty, shock, conflict, and respond conservatively without requiring perfect predictions or calibrated probabilities. This capability is particularly relevant in open-world deployments, where distributional assumptions are routinely violated.

\subsection{Differentiable Extensions}
The current formulation of RLCS is intentionally forward-only. Sensors operate as fixed, deterministic functions evaluated at inference time. This design preserves auditability and prevents reliability logic from becoming entangled with task optimization. A natural extension is \textbf{Differentiable RLCS}, in which sensor outputs contribute to the training objective:
\begin{equation}
    \mathcal{L}_{\text{total}} = \mathcal{L}_{\text{task}} + \lambda \, \mathcal{L}_{\text{reliability}}.
\end{equation}
Such an approach could encourage encoders to learn representations that are not only predictive, but also stable, consistent, and easier to monitor. However, this extension raises important architectural questions. If sensing signals influence learning, care must be taken to preserve the core RLCS invariants, particularly the separation between sensing and acting. Reliability signals may guide representation formation, but they must not collapse into implicit control policies encoded in the model weights.

\subsection{System-Level Implications}
Beyond individual models, RLCS suggests a broader perspective on ML system design. Reliability sensing can be standardized across pipelines, reused across domains, and governed independently of model retraining cycles. This opens the possibility of shared reliability infrastructure, where multiple models or modalities are supervised by a common control surface. In such systems, reliability becomes a first-class interface between learning components and operational policy.

\subsection{Open Questions}
Several open questions remain:
\begin{itemize}
    \item How should reference statistics be maintained under long-term deployment and evolving data distributions?
    \item What are the trade-offs between sensitivity and false positives in different application domains?
    \item How can RLCS-style observability be extended to generative models and long-horizon decision processes?
\end{itemize}
These questions are inherently systems-oriented and may not admit purely algorithmic solutions. RLCS provides a framework within which they can be addressed explicitly rather than implicitly.

\subsection{Perspective}
RLCS does not aim to replace robust training, uncertainty estimation, or safety engineering. Instead, it complements them by making representation reliability observable, interpretable, and actionable at runtime. The broader perspective is simple: as learning systems become larger and more complex, reliability must move out of the weights and into the architecture. RLCS represents one step toward that goal.

\section{Conclusion}

This paper has presented \textbf{Representation-Level Control Surfaces (RLCS)}, a paradigm that challenges the assumption that reliability is an emergent property of learning. This work demonstrated that by externalizing reliability sensing into independent, composable units, ResLik, TCS, and Agreement, we can construct systems that are robust to the silent failures of deep learning. Reliability, ultimately, is a system property, and RLCS provides the architectural blueprint to manage it.

\bibliographystyle{plainnat}
\bibliography{references}

\end{document}
